\documentclass[compress]{beamer}

\usetheme{Hamburg}

\usepackage[utf8]{inputenc}
\usepackage{units}

\title{IVC - Zwischenpräsentation}
\author{Thorben Harms, Felix Ortmann \& Kai Hildebrandt}
\institute{Fachbereich Informatik\\Fakultät für Mathematik, Informatik und Naturwissenschaften\\Universität Hamburg}
\date{18.12.2013}

\titlegraphic{\includegraphics[width=0.5\textwidth]{logo}}

\begin{document}

\begin{frame}
	\titlepage
\end{frame}

\begin{frame}
	\frametitle{Gliederung}

	\tableofcontents[hidesubsections]
\end{frame}

\section{Idee}

\begin{frame}
	\frametitle{Idee}

	\begin{itemize}
	  \item<+-> Ein Blatt Papier ist zu sehen, auf das ein Strichmännchen gemalt ist.
	  \item<+-> Eine Windhose saugt das Strichmännchen "aus dem" Papier in eine dreidimensionale Welt gesogen.
	  \item<+-> Dort passieren Dinge.
	  \item<+-> Das Strichmännchen fällt zurück "in" das Papier und liegt anders dort, als zu Beginn.
	  \item<+-> Außerdem ist ein Würfel mit auf das Blatt Papier gefallen, der allerdings nicht "ins Papier" übergeht, sondern materialisiert auf dem Tisch liegt
	\end{itemize}
\end{frame}

\section{Aktueller Stand}

\begin{frame}
	\frametitle{Aktueller Stand}
	
	\begin{itemize}
	  \item verschiedene Objekte erstellt
	  \begin{itemize}
	    \item Strichmännchen
	    \item Windhose
	    \item Würfel
	    \item Tisch
	    \item etc.
	  \end{itemize}
	\end{itemize}
	
	\begin{figure}
		\begin{center}
			\includegraphics[width=0.75\textwidth]{logo.jpg}
		\end{center}
		\caption{Beispielgrafik}
		\label{fig:logo}
	\end{figure}

\end{frame}

\end{document}
